This study reveals the central role of spatial configuration in shaping flood resilience. In cities like Chennai, where low-lying terrain and fragmented hydrological systems intersect with rapid urban growth, extreme flood events not only damage physical infrastructure but rewire the lived geography of access and isolation. Following Hillier (1987), space is not a passive container of activity but an active generator of urban opportunity. The structure of the street network influences who moves, who connects, and who remains reachable. Yet, as Lavell has long argued, disasters are not simply triggered by nature---they are socially constructed through layers of historical inequality, institutional neglect, and uneven exposure (2003). These two perspectives, when placed in conversation, expose a critical tension: the spatial logic that enables movement can also reproduce exclusion, especially when environmental shocks strike fragile systems. A street network that appears well-integrated in abstraction may fail under stress if it overlays fragmented governance, marginalization, or infrastructural decay. 

This research proposes a synthesis: resilience must be understood as both spatial continuity and social correction. It is not enough to preserve accessibility. The question is who stays connected, and under what conditions. Urban form and social vulnerability co-produce risk---and must coinform adaptation. The methodology developed here, grounded in space syntax, enriched with environmental and infrastructural overlays, and sharpened by socio-spatial critique, offers a replicable tool for diagnosing urban fragility. Its value lies not only in modelling accessibility loss, but in making visible the silent geographies of exclusion that disasters expose. Resilience, then, is not a technical fix. It is a spatial and political project. To plan for floods is to plan for equity---to recognize that how cities are shaped determines who is saved.
