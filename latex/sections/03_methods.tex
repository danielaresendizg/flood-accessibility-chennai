\begin{figure}[ht]
  \centering
  % TikZ diagram only (for inclusion in papers).
\begin{tikzpicture}[
  font=\fontfamily{phv}\selectfont,
  >=Latex,
  node distance=6mm and 10mm,
  stageLabel/.style={font=\small, align=left},
  bar/.style={fill=black, rounded corners=2pt, text=white, font=\small\bfseries, inner xsep=8pt, inner ysep=6pt, minimum width=6.2cm, align=center},
  box/.style={draw, rounded corners=2pt, fill=white, font=\scriptsize, inner xsep=6pt, inner ysep=5pt, align=center, minimum width=3.2cm},
  boxWide/.style={box, minimum width=4.0cm},
  outbox/.style={draw, rounded corners=2pt, fill=gray!20, font=\scriptsize\bfseries, inner xsep=6pt, inner ysep=6pt, align=center, minimum width=4.0cm},
  outSmall/.style={outbox, font=\scriptsize, minimum width=3.2cm},
  arrow/.style={-Latex, line width=0.5pt},
]

% --- Left column (Stages 1–2) ---
\node[stageLabel] (s1lbl) at (0, 3.5) {stage 1};
\node[bar, below=1mm of s1lbl, minimum width=3.6cm] (s1bar) {floods impact on spatial\\configuration};
\node[box, below=of s1bar] (s1a) {normal conditions\\street network\\r3000 / r800};
\node[box, below=of s1a] (s1b) {flood model (100-year return):\\segment deletion and penalty\\levels};
\node[box, below=of s1b] (s1c) {street network under flood conditions\\r3000 / r800};

\node[stageLabel, below=14mm of s1c] (s2lbl) {stage 2};
\node[bar, below=1mm of s2lbl, minimum width=3.6cm] (s2bar) {urban systems and services\\exposure};
\node[box, below=of s2bar, minimum width=3.6cm] (s2a) {land use};
\node[box, below=of s2a, minimum width=3.6cm] (s2b) {health};
\node[box, below=of s2b, minimum width=3.6cm] (s2c) {education};
\node[box, below=of s2c, minimum width=3.6cm] (s2d) {slums};
\node[box, below=of s2d, minimum width=3.6cm] (s2e) {shelters};

% --- Right column (Stages 3–4) ---
\begin{scope}[xshift=7.8cm]
  \node[stageLabel] (s3lbl) at (0, 3.5) {stage 3};
  \node[bar] (s3bar) at (1.9, 3.05) {global-scale resilience strategies};

  % Stage 3 left stack
  \node[boxWide, below=8mm of s3bar, xshift=-2.15cm] (g1a) {population density};
  \node[boxWide, below=of g1a] (g1b) {nachr3000m (core 20\%)\\normal condition};
  \node[boxWide, below=of g1b] (g1c) {nachr3000m\\flood condition};
  \node[boxWide, below=of g1c] (g1d) {critical infrastructure};
  \node[outSmall, below=of g1d] (g1o) {priority emergency\\corridors\\\vspace{1mm}\\structural resilience\\strategies in transport\\infrastructure};

  % Stage 3 middle stack
  \node[boxWide, right=12mm of g1a] (g2a) {Dem (copernicus)};
  \node[boxWide, below=of g2a] (g2b) {waterways network};
  \node[boxWide, below=of g2b] (g2c) {high risk zones};
  \node[outSmall, below=of g2c] (g2o) {designated zones for\\water-sensitive urban\\design (WSD)};

  % Stage 3 right stack
  \node[boxWide, right=12mm of g2a] (g3a) {waterways network};
  \node[boxWide, below=of g3a] (g3b) {high risk streets\\network};
  \node[outSmall, below=of g3b] (g3o) {waterway and street\\network during extreme\\events};

  % connectors stage 3
  \draw[arrow] (g1a) -- (g1b);
  \draw[arrow] (g1b) -- (g1c);
  \draw[arrow] (g1c) -- (g1d);
  \draw[arrow] (g1d) -- (g1o);

  \draw[arrow] (g2a) -- (g2b);
  \draw[arrow] (g2b) -- (g2c);
  \draw[arrow] (g2c) -- (g2o);

  \draw[arrow] (g3a) -- (g3b);
  \draw[arrow] (g3b) -- (g3o);

  % Stage 4 (placed well below Stage 3 to avoid overlaps)
  \node[stageLabel] (s4lbl) at (0, -6.2) {stage 4};
  \node[bar] (s4bar) at (1.9, -6.65) {local-scale resilience strategies};

  \node[boxWide, below=8mm of s4bar, xshift=-2.15cm] (l1a) {population density};
  \node[boxWide, below=of l1a] (l1b) {nachr3000m\\flood condition};
  \node[boxWide, below=of l1b] (l1c) {nachr800m\\flood condition};
  \node[boxWide, below=of l1c] (l1d) {10 clusters (k-means)};
  \node[outSmall, below=of l1d] (l1o) {assessment of shelter\\placement\\\vspace{1mm}\\identification of safe-\\accessibility priority\\zones};

  \node[boxWide, right=12mm of l1a] (l2a) {waterway and street\\network during extreme\\events};
  \node[boxWide, below=of l2a] (l2b) {deployment of emergency\\boats};
  \node[outSmall, below=of l2b] (l2o) {assessment of boat\\deployment efficiency\\under flood conditions};

  \draw[arrow] (l1a) -- (l1b);
  \draw[arrow] (l1b) -- (l1c);
  \draw[arrow] (l1c) -- (l1d);
  \draw[arrow] (l1d) -- (l1o);

  \draw[arrow] (l2a) -- (l2b);
  \draw[arrow] (l2b) -- (l2o);
\end{scope}

% --- Decorative bracket between left and right blocks ---
\draw[line width=0.7pt] (3.4, 3.1) -- (3.4, -8.6);
\draw[line width=0.7pt] (3.4, 3.1) .. controls (3.8, 3.1) and (3.9, 3.1) .. (4.2, 3.1);
\draw[line width=0.7pt] (3.4, -8.6) .. controls (3.8, -8.6) and (3.9, -8.6) .. (4.2, -8.6);

\end{tikzpicture}

  \caption{Methodological framework for flood-accessibility modelling.}
\end{figure}

Understanding the interaction between floods and urban spatial configuration is critical to developing resilient cities under increasing climate threats. This study focuses on the territory of Chennai and employs a multi-scalar, systemic approach to model the impacts of extreme flood events on urban networks. It proposes targeted resilience strategies at both global and local levels (Figure 2). To assess the reshaping of spatial structures under flood conditions, street network accessibility is contrasted in normal and flood scenarios. Two scales are considered: a broader accessibility radius (r3000) and a localized radius (r800). The normal network is evaluated against a 100-year return period flood model (OpenCity Data), generating penalty classifications: high (segments removed), moderate (75\% penalty), and low (25\% penalty). Flood overlays on the street network allow quantification of accessibility degradation and spatial disruptions caused by floodwaters. The reconfigured network is analysed to identify cascading effects on critical urban systems. Accessibility is evaluated across five domains: land use, health services, education infrastructure, informal settlements (slums), and emergency shelters \\citep{gcc2024}. Severity is categorized using a four-tier scale, ranging from high to null, based on the 100-year flood model. This enables spatial prioritization of interventions and reveals patterns of vulnerability. Resilience strategies are structured at both global and local scales. At the global level, population density, critical infrastructure, and baseline accessibility (top 20\% of NACHr3000) form the foundational layers. Flood-adjusted accessibility (NACHr3000) is analysed to identify key street and bridge segments requiring intervention. Emergency response routes are delineated by merging NACHr3000 under normal and flood conditions. These routes inform resilience measures using the Climate App (DELTARES platform). A Digital Elevation Model (Copernicus GLO-30 DEM) and waterway data (India Water Resources System) support identification of flood-prone zones. These are integrated with street network data to guide the application of water-sensitive design strategies. A complementary model combining water networks and flood-impacted accessibility is developed to support evaluations under extreme conditions. Although created for local analysis, this model remains available for future applications, including hybrid mobility strategies. At the local level, population density, flood-adjusted accessibility (NAINr800 and NAINr3000), and Google's Open Buildings dataset are aggregated at the block level. All data is normalized between 0 and 1 to ensure comparability. The clustering model is applied only in zones free from flooding, with the goal of identifying the safest areas. K-means clustering (k=10) prioritizes population density, followed by global and local accessibility. This enables localized strategic planning. Clustering results support evaluation of shelter locations (Disaster Risk Reduction Plan, GCC). Analysis confirms which emergency shelters maintain accessibility under flood conditions. Priority blocks are identified for future shelter allocation based on combined network and risk modeling. Based on the global waterways and impacted network model, a local-scale assessment of emergency boat placement (GCC) is conducted. The model evaluates their spatial alignment with local accessibility during flood events and identifies opportunities to enhance hybrid evacuation systems. The model remains adaptable for future mobility planning. This methodology culminates in an integrated framework that links normal and flood-condition accessibility with socio-infrastructural vulnerabilities. By combining global-scale interventions--- such as route prioritization---with localized analyses of shelter and boat-based accessibility, the framework delivers a comprehensive resilience strategy. It balances immediate emergency needs with long-term adaptive planning and offers a scalable model for cities confronting hydro-climatic disruption.
