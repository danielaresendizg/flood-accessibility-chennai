Understanding flood resilience requires frameworks that account for both environmental systems and the structure of urban space. Two influential paradigms---Water Sensitive Urban Design (WSUD) and Space Syntax---have shaped how cities are conceptualized and reconfigured in response to climate risk. Yet despite their strengths, both tend to operate in silos, and few studies have sought to merge them into a cohesive analytical model, particularly in contexts like Chennai, where infrastructure and environmental inequality intersect at critical thresholds. WSUD emphasizes the ecological reengineering of urban landscapes through localized, sustainable interventions: permeable surfaces, bio-retention, and decentralized stormwater systems (Wong \& Brown, 2011). These strategies are now central to resilience discourse, particularly in Global South cities facing compound risks of climate exposure and infrastructure fragility. However, even the most advanced WSUD applications---such as flood-sensitive subcatchment modelling (Wu et al., 2023)---are primarily hydraulic in orientation. They rarely address the spatial realities of urban mobility or accessibility loss. Streets are treated as drainage conduits or ecological buffers, not as vital connectors in times of crisis. As a result, the capacity of WSUD to engage with dynamic, lived experiences of isolation or exclusion during flood events remains limited. Space Syntax, conversely, places spatial structure at the centre of analysis. It offers powerful tools to model movement, visibility, and integration across urban networks (Hillier \& Hanson, 1984). Its application to disaster scenarios---such as flood impact mapping (Gil \& Steinbach, 2008) or post-tsunami evacuation modelling (Maureira \& Karimi, 2017)---has demonstrated how changes in accessibility can amplify vulnerability. Yet even here, the logic of the model remains largely spatial and topological. Environmental processes such as water flow or infrastructural degradation are often treated as external disruptions rather than as co-constitutive forces. Models typically assume a stable street network, abstracted from the fluid, temporal disruptions that disasters produce. This reveals a double-blind spot: WSUD neglects the dynamic realities of spatial access, while Space Syntax omits the physical disruptions wrought by environmental change. Neither, on their own, can fully describe what it means to become disconnected---from services, safety, or opportunity---during a flood.



Few existing studies address this dual gap. Dynamic modelling of accessibility loss---where routes disappear, and shelters or hospitals shift from reachable to unreachable---remains largely unaccounted for. This is particularly striking given that resilience is, fundamentally, a question of connection: who remains linked, and who is left behind? This study contributes to this emerging space by proposing a hybrid framework that integrates the environmental sensitivities of WSUD with the configurational insight of Space Syntax. By simulating accessibility degradation under a 100-year flood model and embedding it in the spatial morphology of Chennai, the study reframes vulnerability not just as exposure, but as severed accessibility. The 100-year return period was selected based on its widespread use in hydraulic infrastructure design and urban risk planning, where it defines a threshold for low-probability, highimpact events (FEMA, 2023; IPCC, 2022). This ensures that the model captures not only the most disruptive flood scenarios, but also those most relevant for long-term strategic interventions. In doing so, the framework offers a more situated, network-based understanding of resilience---one that recognizes space not as a static backdrop to climate events, but as a structure in flux, capable of amplifying or absorbing systemic stress.
