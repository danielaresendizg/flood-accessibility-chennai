According to the World Meteorological Organization, floods are the deadliest natural hazards, striking numerous regions worldwide each year and causing over \$40 billion in damages annually \\citep{wmo2023}. The United Nations Office for Disaster Risk Reduction reports that since 1980, there have been 4,588 flood disasters across 172 countries, resulting in more than 250,000 deaths and over \$1 trillion in damages, accounting for 40\% of natural catastrophe losses during that period \\citep{marshmclennan2021}. Furthermore, the World Bank highlights that 1.81 billion people---approximately 23\% of the global population---reside in areas directly exposed to 1-in-100-year flooding events, with 89\% of these individuals living in low- and middle-income countries \\citep{worldbank2022}. Recent estimates reinforce the urgency of addressing urban flood vulnerabilities. South and East Asia are particularly burdened, with China and India alone comprising over one-third of global flood exposure, and India accounting for 390 million exposed individuals \\citep{rentschler2022}. Urbanization patterns significantly exacerbate flood risks. As cities expand, impermeable surfaces proliferate, natural drainage systems are disrupted, and informal settlements increasingly encroach upon high-risk floodplains. This rapid, often unplanned growth not only amplifies exposure but also strains existing infrastructure, particularly in emerging megacities across the Global South. Among these, Chennai stands as a critical case study. Historically shaped by colonial expansion, Chennai's urban growth evolved along its intricate network of rivers and waterways, including the Cooum, Adyar, and Kosasthalaiyar rivers, which once cradled the city with fertile floodplains and vital navigation routes. These waterways created a landscape where canals intertwined with settlement clusters, resembling the interconnected urban-water systems described by early chroniclers of other great cities, evoking a vivid image of a city deeply embedded within its natural hydrography. From an initial footprint of 1.6 km² in 1633, the city expanded dramatically to over 328 km² by 2006 (Figure 1), reflecting successive waves of territorial consolidation and economic transformation. This expansion, while fueling economic growth, progressively altered natural drainage patterns and intensified flood vulnerability. The colonial urban grid, initially organized 

around port access and administrative centres, eventually fragmented into a sprawling metropolitan area, with many new developments encroaching upon low-lying and hydrologically sensitive zones. Today, Chennai's vulnerability to both pluvial and coastal flooding is closely intertwined with this historical trajectory of urbanization, compounded by recent decades of unregulated land use change and the degradation of its natural water systems. The severance of the once-symbiotic relationship between the urban fabric and its waterways has diminished the city's natural flood absorption capacity, transforming former floodplains and canal corridors into impermeable, fragmented landscapes increasingly susceptible to hydroclimatic extremes. This study examines Chennai to address three central research questions: (1) How do floods reshape the spatial accessibility of urban street networks? (2) How does the reconfigured urban structure affect critical systems such as health services, education, land use, slums, and shelters? and (3) How can resilience be improved at both global and local scales by strategically enhancing accessibility to critical services during flood events? The underlying hypothesis is that flooding significantly degrades urban accessibility, particularly affecting critical services, and that urban spatial configuration either mitigates or exacerbates these impacts. To explore these questions, the research integrates spatial analysis methodologies---specifically Space Syntax techniques---with environmental vulnerability assessments that include flood modelling, land elevation data, hydrological mapping, soil characteristics analysis, water body identification, urban infrastructure, and critical facilities mapping, along with land use classification. This interdisciplinary approach moves beyond traditional hazard mapping to incorporate the dynamics of human movement and network resilience, offering a more comprehensive understanding of urban vulnerability. Chennai's case further exemplifies structural challenges common to Global South cities: systemic urban governance deficiencies, rapid demographic expansion, chronic data scarcity, and entrenched environmental degradation. These conditions reveal that many so-called 'natural disasters' are, in fact, manifestations of deep-rooted socio-spatial vulnerabilities rather than purely environmental phenomena. This perspective aligns with Lavell's argument that disasters are socially constructed outcomes of pre-existing vulnerabilities embedded within territorial and urban systems \\citep{lavell2003}. Notably, the analysis leverages open-source and satellite-derived data to construct diagnostic models, acknowledging that governmental datasets are often outdated, incomplete, or inaccessible. This methodological choice not only addresses immediate data gaps 

but also proposes a replicable framework for similar contexts where resource constraints hinder extensive primary data collection. By synthesizing spatial configuration analysis with environmental risk factors, this research contributes to emerging discussions around nature-based solutions, hybrid mobility systems, and multi-hazard resilience planning. Ultimately, it aims to provide actionable insights that can guide adaptive urban planning, optimize emergency resource allocation, and inform strategic interventions in flood-prone urban environments. In doing so, the study situates itself at the intersection of urban morphology, environmental science, and disaster resilience, seeking to advance methodologies capable of supporting climate-adaptive urban transformations in the 21st century. Building on this foundation, the following section reviews the existing literature on spatial configuration analysis and environmental resilience, identifying critical gaps that this study aims to address.

\begin{figure}[h]
\centering
\includegraphics[width=\textwidth]{figures/urban_growth_chennai.jpg}
\caption{Urban expansion of Chennai (1633--2006). \textit{Source:} Greater Chennai Corporation (GCC)}
\label{fig:urban_growth}
\end{figure}
