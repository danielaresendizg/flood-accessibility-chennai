Flooding is a rapidly intensifying urban challenge that disrupts essential services and exposes structural inequalities. While many resilience frameworks focus on exposure or engineered protection, fewer capture how flood events dynamically alter everyday accessibility---disconnecting populations from shelters, services, and safe mobility routes.

This paper reframes resilience as the preservation of spatial connection under disruption. Using Chennai, India as a case study, we quantify accessibility loss under an extreme flood scenario and identify spatial priorities for evacuation and shelter planning.
