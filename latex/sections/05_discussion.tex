The findings of this study underscore the role of urban spatial configuration in modulating Chennai's resilience to extreme flood events. Accessibility losses across different radii (r3000 and r800) confirm that not only the physical extent of floods but also the systemic organization of the street network critically shape impacts. Broad-scale accessibility (NAINr3000) experienced significant decline under flooding, while localized accessibility (NACHr800) retained partial functionality, highlighting differentiated vulnerabilities across scales. Compact urban forms, such as Cluster 7, displayed resilience by concentrating accessibility, while the broader dispersion of Cluster 9 emphasized the importance of redundancy across the urban system. The strategic design of emergency priority routes ensured the accessibility of key infrastructures, such as hospitals, the airport, and the port. Focusing on the top 20\% of accessible corridors under flood scenarios illustrates that selective infrastructural reinforcement can preserve critical urban functionality. Moreover, analysing emergency boat deployment zones introduced the need for hybrid mobility systems. The uneven distribution of boats in Chennai highlights critical gaps in flood-time accessibility, reinforcing that diversified evacuation options---across land and water--- are vital to urban resilience. Cross-referencing elevation, flood risk, land use, and infrastructure data revealed strong spatial coincidence between environmental vulnerability and socio-economic density. Nature-based interventions, such as floodplain excavations and density management strategies, emerge as essential for mitigating risk. Integrating spatial and environmental logics advances WaterSensitive Urban Design (WSUD) principles, ensuring interventions align with natural hydrological patterns. Restoring Chennai's historic relationship with its waterways and floodplains is critical to reducing systemic flood risk and reestablishing the city's natural capacity for resilience. The analysis confirms that Chennai's exposure to floods stems from deep-seated structural vulnerabilities: rapid demographic growth, governance deficiencies, data scarcity, and environmental degradation. These findings align with Lavell's (2003) argument that disasters are not natural, but the result of accumulated socio-spatial vulnerabilities embedded within territorial and urban systems. Recognizing disasters as socially constructed reframes resilience from a reactive posture toward proactive, structural urban adaptation, addressing the root causes rather than merely the symptoms of vulnerability. 

Several limitations must be acknowledged. The accessibility analysis relies on static flood modelling based on a 100-year return period, whereas dynamic, multi-scenario modelling could capture temporal variability more accurately. Additionally, although this study integrates population density and land use, detailed demographic vulnerability metrics (such as age, disability, or income) could not be fully incorporated due to data limitations, notably following Chennai's administrative restructuring from 150 to 200 wards post-2011. Future research could enhance this model by incorporating social vulnerability indices and simulating real-time accessibility shifts under flood conditions. Hybrid mobility systems could also be expanded by exploring autonomous watercraft and amphibious evacuation strategies. This study advances a multi-scalar, interdisciplinary framework that synthesizes spatial configuration analysis with environmental vulnerability assessments to guide urban resilience strategies. Prioritizing accessibility reinforcement, integrating nature-based interventions, and recognizing the structural nature of urban risks offer a replicable model for cities in the Global South confronting intensifying hydro-climatic threats. By highlighting systemic vulnerabilities and proposing hybrid resilience solutions, this work contributes to building cities that not only survive but adapt to future extremes.
